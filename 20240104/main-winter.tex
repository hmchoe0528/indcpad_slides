\documentclass{beamer}
\usetheme{metropolis}
\usepackage[ruled, lined, linesnumbered, commentsnumbered]{algorithm2e}
\usepackage{fourier}
\usepackage{amssymb}
\usepackage{pifont}% http://ctan.org/pkg/pifont
\newcommand{\cmark}{\ding{51}}%
\newcommand{\xmark}{\ding{55}}%\usepackage{amsmath}
\usepackage{tikz,balance,multicol,multirow}
% for hdashline
\usepackage{arydshln}


\usetikzlibrary{calc, shapes, backgrounds, positioning, arrows, decorations, fit}
\usepackage{pgfplots}
\newcommand{\av}{\mathbf{a}}
\newcommand{\cv}{\mathbf{c}}
\newcommand{\dv}{\mathbf{d}}
\newcommand{\ev}{\mathbf{e}}
\newcommand{\fv}{\mathbf{f}}
\newcommand{\rv}{\mathbf{r}}
\newcommand{\sv}{\mathbf{s}}
\newcommand{\tv}{\mathbf{t}}
\newcommand{\uv}{\mathbf{u}}
% \renewcommand{\vv}{\mathbf{v}}
\newcommand{\wv}{\mathbf{w}}
\newcommand{\xv}{\mathbf{x}}
\newcommand{\yv}{\mathbf{y}}
\newcommand{\zv}{\mathbf{z}}
\renewcommand{\AA}{\mathbf{A}}
\newcommand{\BB}{\mathbf{B}}
\newcommand{\CC}{\mathbf{C}}
\newcommand{\FF}{\mathbf{F}}
\newcommand{\GG}{\mathbf{G}}
\newcommand{\RR}{\mathbf{R}}
\newcommand{\Sb}{\mathbf{S}}
%\renewcommand{\SS}{\mathbf{S}}
\newcommand{\TT}{\mathbf{T}}
\newcommand{\UU}{\mathbf{U}}
\newcommand{\VV}{\mathbf{V}}
\newcommand{\WW}{\mathbf{W}}
\newcommand{\XX}{\mathbf{X}}
\newcommand{\ZZ}{\mathbf{Z}}
\newcommand{\II}{\mathbf{I}}

\newcommand{\SD}{\mathsf{SD}}
\newcommand{\DI}{\mathsf{DI}}
\newcommand{\TM}{\mathsf{TM}}
\newcommand{\mife}{{\mathsf {miFE}}}
\newcommand{\MIFE}{\mife}
\newcommand{\disfe}{\mathsf{DiFE}}%\newcommand{\tmmife}{{\mathsf{TM\mbox{-}miFE}}}
\newcommand{\sfe}{\mathsf{1FE}}
\newcommand{\Sym}{{\sf Sym}}
\newcommand{\val}{{\sf val}}
\newcommand{\tmmife}{{\mathsf{kTMFE}}}
\newcommand{\FELin}{\mathsf{LinFE}}
\newcommand{\FEPoly}{\mathsf{PolyFE}}
\newcommand{\PolyFE}{\FEPoly}
\newcommand{\NFELin}{\mathsf{NLinFE}}
\newcommand{\FEQuad}{\mathsf{QuadFE}}
\newcommand{\sfU}{\mathsf{U}} %Universal TM
\newcommand{\sfenc}{\mathsf{enc}}
\newcommand{\fxdmife}{\mathsf{(n\!+\!1)\text{-}TMFE}}


%%% For branching program %%%%
\newcommand{\BP}{\mathsf{BP}}
\newcommand{\var}{\mathsf{var}}
\newcommand{\ToDFA}{\mathsf{ToDFA}}
\newcommand{\encinp}{\mathsf{\encode}}
\newcommand{\ed}{\mathsf{ed}} % \end already defined
%%% For random variables %%%%

\def \rd {\mathfrak{d}}
\def \rF {\mathfrak{F}}
\def \rD {\mathfrak{D}}
\def \rU {\mathfrak{U}}
\def \rc {\mathfrak{c}}
\def \rgamma {\mathfrak{\gamma}}
\def \rmu {\mathfrak{\mu}}

%%% For adversaries and games %%%
\newcommand{\aA}{\mathsf{A}}
\newcommand{\aB}{\mathsf{B}}
\newcommand{\game}{\mathbf{Game}}
\newcommand{\Event}{\mathsf{E}}
\newcommand{\aP}{\mathsf{P}}
\newcommand{\aV}{\mathsf{V}}
\newcommand{\aR}{\mathsf{R}}
\newcommand{\redunderline}[1]{\textcolor{red}{\underline{\textcolor{black}{#1}}}}

\newcommand{\ul}[1]{\ensuremath{\underline{#1}}}

%%% For product vectors and matrices %%%

\def\cA{{\mathcal A}}
\def\cB{{\mathcal B}}
\def\cC{{\mathcal C}}
\def\cD{{\mathcal D}}
\def\cE{{\mathcal E}}
\def\cF{{\mathcal F}}
\def\cG{{\mathcal G}}
\def\cH{{\mathcal H}}
\def\cI{{\mathcal I}}
\def\cJ{{\mathcal J}}
\def\cK{{\mathcal K}}
\def\cL{{\mathcal L}}
\def\cM{{\mathcal M}}
\def\cN{{\mathcal N}}
\def\cO{{\mathcal O}}
\def\cP{{\mathcal P}}
\def\cQ{{\mathcal Q}}
\def\cR{{\mathcal R}}
\def\cS{{\mathcal S}}
\def\cT{{\mathcal T}}
\def\cU{{\mathcal U}}
\def\cV{{\mathcal V}}
\def\cW{{\mathcal W}}
\def\cX{{\mathcal X}}
\def\cY{{\mathcal Y}}
\def\cZ{{\mathcal Z}}
\def\bK{\mathbf {K}}
\def\cfam{\mathcal{F}}
\def\dfam{\mathcal{M}}
\def\N{\mathbb{N}}
\def\fA{\mathbf{A}}
\def\one{\mathbbm{1}}
\newcommand{\fename}{\mathsf{FE}}
\newcommand{\iO}{\mathsf{iO}}
\newcommand{\SKE}{\mathsf{SKE}}
%------------------------------
%\newcommand{\secp}{\kappa}
%Changed on 4.9.17
\newcommand{\secp}{\lambda}
%------------------------------
\newcommand{\secparam}{\secp}
\newcommand{\noise}{\chi}
\newcommand{\noiseone}{\chi_{\alpha_1}}
\newcommand{\noisetwo}{\chi_{\alpha_2}}
\newcommand{\noisei}{\chi_{\alpha_i}}
\newcommand{\noised}{\chi_{\alpha_d}}

\newcommand{\noiseonet}{\chi_{\tilde\alpha_1}}
\newcommand{\noisetwot}{\chi_{\tilde\alpha_2}}
\newcommand{\noiseit}{\chi_{\tilde\alpha_i}}
\newcommand{\noisedt}{\chi_{\tilde\alpha_d}}

\newcommand{\noiset}{\chi_{\tilde \alpha}}
\newcommand{\noiseal}{\chi_\alpha}

% %\newif\iflncs
% %	\lncstrue
% % comment out above line to get full version

% \newif\ifext
% %   \exttrue
% % uncomment the above line to get the "extended" version with
% % conjunctions of inner products and more...

% %\iflncs
% %\usepackage{breakcites}
% %\else
% %
% 	%\usepackage[hmargin=1in,vmargin=1.25in]{geometry}
% %	\usepackage{amsthm}
%   % \usepackage{xcolor}
% 	%\usepackage[pdfstartview=FitH,colorlinks,linkcolor=blue,citecolor=blue]{hyperref}

% %\fi

% 	%\usepackage[affil-it]{authblk}
% 	%\usepackage[hmargin=0.8in,vmargin=0.8in]{geometry}
% 	%\usepackage [margin=1 in]{geometry}
% 	%\usepackage{amsthm}
	
% % 	\documentclass[orivec, envcountsame, envcountsect]{llncs}
%      \usepackage{breakcites}
% %	\usepackage{enumlist}
%     \usepackage{xcolor}
% %	\usepackage[pdfstartview=FitH,colorlinks,linkcolor=blue,citecolor=blue]{hyperref}
% %	\usepackage{fullpage}
% 	\usepackage{tabu}
% 	\newcommand\hmmax{0}
% 	\newcommand\bmmax{0}
% 	\usepackage{caption}
% 	\DeclareCaptionLabelSeparator{lsep}{ }
% 	\captionsetup{labelsep=lsep}

%\usepackage{amsfonts, amsthm, amsmath, amssymb}
%\usepackage{fullpage,caption,subcaption,enumitem,array}
%\usepackage[T1]{fontenc}
%\usepackage[latin9]{inputenc}
%\usepackage{enumitem}
\usepackage{tikz,balance}
\usetikzlibrary{calc, shapes, backgrounds, positioning, arrows, decorations, fit}
\usepackage{microtype}
\usepackage{amsmath,amsfonts,mathrsfs}
%\usepackage{amssymb}
\usepackage{url}
\usepackage{xspace,paralist}
\usepackage{nicefrac}
\usepackage{algorithmicx}
\usepackage{framed}
\usepackage{graphicx}
\usepackage{breakcites}
\usepackage{mathtools}
\numberwithin{equation}{section}
\usepackage{xcolor}
\usepackage{caption} 
\captionsetup[table]{skip=5pt}
%\usepackage{placeins}
\usepackage[]{algorithm2e}

%\pagenumbering{gobble}

% THEOREMS %%%%%%%%%%%%%%%%%%%%%%%%%%%%%%%%%%%%%%%%%%%%%%%%%%%%%%%%%%%%%%%%%%%
%
%\iflncs
%\spnewtheorem{defn}[theorem]{Definition}{\bfseries}{\rmfamily}
%\spnewtheorem{rem}[theorem]{Remark}{\bfseries}{\rmfamily}
%\else

% % Theorem definitions
% \theoremstyle{plain}
 %\newtheorem{theorem}{Theorem}[section]
% \newtheorem{lemma}[theorem]{Lemma}
% \newtheorem{proposition}[theorem]{Proposition}
% \newtheorem{corollary}[theorem]{Corollary}
% \newtheorem{fact}[theorem]{Fact}
% \newtheorem{hypothesis}[theorem]{Hypothesis}
\newtheorem{numclaim}[theorem]{Claim}

% \theoremstyle{definition}
% \newtheorem{definition}[theorem]{Definition}
% \newtheorem{rem}[theorem]{Remark}
% \newtheorem{alg}[theorem]{Algorithm}
% \newcommand{\inst}[1]{}

% \theoremstyle{remark}
% \newtheorem{remark}[theorem]{Remark}
% \newtheorem{example}[theorem]{Example}

%\fi



\newcommand{\annote}[2]{\par\bigskip\noindent{\Huge(}\quad\fbox{\sf{#1}}\par\medskip{#2}\smallskip\par\noindent{\Huge)}\bigskip\par}

%\newcommand{\comment}[1]{}
\newcommand{\ignore}[1]{}
\newcommand{\deq}{\mathrel{\mathop:}=}
\newcommand{\nicehalf}{{\nicefrac{1}{2}}}
\newcommand{\half}{{\frac{1}{2}}}
%\newenvironment{alg}{\begin{quote}\begin{tabular}{l}}{\end{tabular}\end{quote}}

\addtolength{\parskip}{3pt}
\hyphenpenalty=5000
\tolerance=1000


\newcommand{\dist}{\Delta}
\newcommand{\noisedist}[1]{\overline\Psi_{#1}}

\newcommand{\chal}{*}
\newcommand{\dual}{*}
\newcommand{\perplat}[1]{\ensuremath{\Lambda_q^{\perp}({#1})}}
\newcommand{\shiftedlat}[2]{\ensuremath{\Lambda_q^{{#2}}({#1})}}
\newcommand{\normR}{s_{\scriptscriptstyle{R}}}
\newcommand{\width}{\text{\rm width}}

\newcommand{\SCA}[1]{\textit{#1}}
\newcommand{\VEC}[1]{\mathbf{#1}}
\newcommand{\MAT}[1]{\mathsf{#1}}

\newcommand{\SAMPLEPRE}{\mathsf{SamplePre}}
\newcommand{\SAMPLEGAUSSIAN}{\mathsf{SampleGaussian}}
\newcommand{\SAMPLEBASIS}{\mathsf{SampleBasis}}
\newcommand{\GENSAMPLEPRE}{\mathsf{GenSamplePre}}
\newcommand{\SAMPLELEFT}{\mathsf{SampleLeft}}
\newcommand{\SAMPLERIGHT}{\mathsf{SampleRight}}
\newcommand{\TRAPGEN}{\mathsf{TrapGen}}
\newcommand{\RANDBASIS}{\mathsf{RandBasis}}
\newcommand{\BASISDEL}{\mathsf{BasisDel}}



\newcommand{\BF}{\mathbf}     
\newcommand{\DDH}{\mathsf{DDH}}  
\newcommand{\DDDH}{\mathsf{D3DH}} 
% \newcommand{\LWE}{\mathsf{LWE}}

% Vectors, Matrices and such
\def\matA{\mathbf{A}}
\def\matT{\mathbf{T}}
\def\matB{\mathbf{B}}
\def\matC{\mathbf{C}}
\def\matD{\mathbf{D}}
\def\matG{\mathbf{G}}
\def\matL{\mathbf{L}}
\def\matV{\mathbf{V}}
\def\matW{\mathbf{W}}
\def\matF{\mathbf{F}}
\def\matH{\mathbf{H}}
\def\matM{\mathbf{M}}
\def\matS{\mathbf{S}}
\def\matR{\mathbf{R}}
\def\matU{\mathbf{U}}
\def\matE{\mathbf{E}}
\def\matK{\mathbf{K}}
\def\matP{\mathbf{P}}
\def\matY{\mathbf{Y}}
\def\matZ{\mathbf{Z}}
\def\matI{\mathbf{I}}


\def\veca{\mathbf{a}}
\def\vecb{\mathbf{b}}
\def\vecc{\mathbf{c}}
\def\vecd{\mathbf{d}}
\def\vecD{\mathbf{D}}
\def\vecE{\mathbf{E}}

\def\vecw{\mathbf{w}}
\def\vece{\mathbf{e}}
\def\vecf{\mathbf{f}}
\def\vecg{\mathbf{g}}
\def\vech{\mathbf{h}}
\def\veck{\mathbf{k}}

\def\veczero{\mathbf{0}}

\def\vecu{\mathbf{u}}
\def\vecr{\mathbf{r}}
\def\vecm{\mathbf{m}}
\def\vecs{\mathbf{s}}
\def\vecj{\mathbf{j}}
\def\vect{{\mathbf{t}}}
\def\vecv{\mathbf{v}}
\def\vecw{\mathbf{w}}
\def\vecx{\mathbf{x}}
\def\vecY{\mathbf{Y}}
\def\vecy{\mathbf{y}}
\def\vecL{\mathbf{L}}
\def\vecz{\mathbf{z}}
\def\veceta{{\boldsymbol{\eta}}}
\def\vecbeta{{\boldsymbol{\beta}}}
\def\vecdelta{{\boldsymbol{\delta}}}
\def\vecmu{{\boldsymbol{\mu}}}
\def\vecnu{{\boldsymbol{\nu}}}
\def\vecpi{{\boldsymbol{\pi}}}
\def\vecphi{{\boldsymbol{\phi}}}
\def\vecpsi{{\boldsymbol{\psi}}}
\def\vectau{{\boldsymbol{\tau}}}
\def\vecrho{{\boldsymbol{\rho}}}
\def\vecgamma{\boldsymbol{\gamma}}
\def\veceps{\boldsymbol{\epsilon}}
\def\halfq{\left\lfloor\frac{q}{2}\right\rfloor}

%%% For Product Objects %%%

\def\pmatE{\mathbf{E}^{\times}}
\def\pvecc{\mathbf{c}^{\times}}
\def\pvecs{\mathbf{s}^{\times}}

\def\pvecgamma{\boldsymbol{\gamma}^\times}


\newcommand{\bits}{\mathsf{BitDec}}
\newcommand{\potwo}{\mathsf{Po2}}
\newcommand{\pad}{\mathsf{pad}}
\newcommand{\padlength}{p}
\newcommand{\ToCircuit}{\mathsf{To\text{-}Circuit}}

\newcommand{\ZQ}{\mathbb{Z}_q}
\newcommand{\Z}{\mathbb{Z}}
\newcommand{\F}{\mathbb{F}}
\newcommand{\G}{\mathbb{G}}
\newcommand{\R}{\mathbb{R}}
\def\Q{\mathbb{Q}}
\def\bbC{{\mathbb C}}
\def\bbE{{\mathbb E}}
\def\bbF{{\mathbb F}}
\def\bbG{{\mathbb G}}
\def\bbM{{\mathbb M}}
\def\bbN{{\mathbb N}}
\def\bbQ{{\mathbb Q}}
\def\bbR{{\mathbb R}}
\def\bbV{{\mathbb V}}
\def\bbZ{{\mathbb Z}}
\def\bbS{{\mathbb S}}

\newcommand{\dsetup}{\textsf{Setup}}
\newcommand{\encode}{\mathsf{Encode}}
\newcommand{\garble}{\textsf{Garble}}

\newcommand{\enc}{\mathsf{Enc}}
\newcommand{\attr}{{\mathsf{attr}}}
\newcommand{\stmt}{{\sf stmt}}

\newcommand{\Gen}{\mathsf{Gen}}
\newcommand{\Vrfy}{\mathsf{Vrfy}}
%\newcommand{\setup}{\mathsf{Setup}}
%\newcommand{\extract}{\mathsf{KeyGen}}
%\newcommand{\encrypt}{\mathsf{Enc}}
%\newcommand{\decrypt}{\mathsf{Dec}}
%\newcommand{\evaluate}{\mathsf{Eval}}
%\newcommand{\keydel}{\mathsf{KeyDel}}
%\newcommand{\constrain}{\mathsf{Constrain}}
%\newcommand{\setup}{\mathsf{Setup}}


\newcommand{\DERIVE}{\textsf{FE.Derive}}
\newcommand{\SAMPLEMAT}{\textsf{SampleRwithBasis}}
\newcommand{\SAMPLELOWMAT}{\textsf{SampleR}}

\newcommand{\LinSETUP}{\textsf{FELin.Setup}}
\newcommand{\LinEXTRACT}{\textsf{FELin.KeyGen}}
\newcommand{\LinENCRYPT}{\textsf{FELin.Enc}}
\newcommand{\LinDECRYPT}{\textsf{FELin.Dec}}


\newcommand{\SimSetup}{\textsf{Sim.Setup}}
\newcommand{\SimEncrypt}{\textsf{Sim.Enc}}
\newcommand{\SimExtract}{\textsf{Sim.KeyGen}}



\newcommand{\FEBd}{\mathsf{BddFE}}
\newcommand{\BdFE}{\mathsf{BddFE}}
\newcommand{\BddSETUP}{\textsf{BddFE.Setup}}
\newcommand{\BddEXTRACT}{\textsf{BddFE.KeyGen}}
\newcommand{\BddENCRYPT}{\textsf{BddFE.Enc}}
\newcommand{\BddDECRYPT}{\textsf{BddFE.Dec}}

\newcommand{\PE}{{\mathsf{PE}}}
\newcommand{\PEp}{{\mathsf{PE}}^{+}}
\newcommand{\ABE}{{\mathsf{ABE}}}
\newcommand{\FHE}{\mathsf{FHE}}
\newcommand{\HE}{\mathsf{FHE}}
\newcommand{\HS}{\mathsf{HS}}
\newcommand{\yao}{\mathsf{Yao}}
\newcommand{\fhelen}{\lambda}
\newcommand{\tor}{{\sf TOR}}
\newcommand{\msg}{\mu}


\newcommand{\HPK}{\mathsf{HPK}}
\newcommand{\HSK}{\mathsf{HSK}}
\newcommand{\FPK}{\mathsf{FPK}}
\newcommand{\FMPK}{\mathsf{FMPK}}
\newcommand{\FMSK}{\mathsf{FMSK}}
\newcommand{\FSK}{\mathsf{FSK}}
\newcommand{\succfe}{\mathsf{SuccFE}}
\newcommand{\GKPVZ}{\mathsf{GKPVZ}}


\newcommand{\RE}{\sf{RE}}
\newcommand{\REnc}{\sf{Encode}}
\newcommand{\RDec}{\sf{Decode}}
\newcommand{\RESim}{\textsf{RE.Sim}}

\newcommand{\gc}{\sf{GC}}
\newcommand{\gin}{\sf{GInp}}
\newcommand{\gcirc}{\sf{GCirc}}
\newcommand{\geval}{\sf{GEval}}
\newcommand{\gsim}{\textsf{GC.Sim}}

\newcommand{\LinSim}{\textsf{LinFE.Sim}}
\newcommand{\PolySim}{\textsf{PolyFE.Sim}}
\newcommand{\BdSim}{\textsf{Bdd.Sim}}

\newcommand{\ASETUP}{\textsf{ABE.Setup}}
\newcommand{\AEXTRACT}{\textsf{ABE.Extract}}
\newcommand{\AENCRYPT}{\textsf{ABE.Enc}}
\newcommand{\ADECRYPT}{\textsf{ABE.Dec}}

\newcommand{\PP}{\ensuremath{\mathsf{PP}}}
\newcommand{\MK}{\ensuremath{\mathsf{MSK}}}
\newcommand{\PK}{\ensuremath{\mathsf{MPK}}}
\newcommand{\EK}{\ensuremath{\mathsf{EK}}}

\newcommand{\ID}{\mathsf{id}}
\newcommand{\I}{\mathsf{I}}
\newcommand{\CT}{\mathsf{ct}}
\newcommand{\sigmaR}{\sigma_{\scriptscriptstyle \mathrm{R}}}
\newcommand{\sigTG}{\sigma_{\scriptscriptstyle \mathrm{TG}}}
	
\def\bydef{\stackrel{.}{=}}
\newcommand{\ext}{\sf{ext}}
\newcommand{\Sch}{\sf{Sch}}


\newcommand{\FOE}{\mathcal{A}}
\newcommand{\SIM}{\mathcal{B}}
\newcommand{\ORA}{\mathcal{O}}
\newcommand{\EXP}[1]{\mathrm{E}\!\big\{{#1}\big\}}
\newcommand{\VAR}[1]{\mathrm{V}\!\big\{{#1}\big\}}
\newcommand{\IDsub}[1]{{#1}_{\scriptscriptstyle{\ID}}}
\newcommand{\privID}{\IDsub{\SK}}
\newcommand{\qs}[1]{Q_{\scriptscriptstyle{#1}}}

\newcommand{\CPIDCPA}{\text{\sf CP-ID-CPA}}
\newcommand{\CPsIDCPA}{\text{\sf CP-sID-CPA}}
\newcommand{\INDIDCCA}{\text{\sf IND-ID-CCA2}}
\newcommand{\INDIDCPA}{\text{\sf IND-ID-CPA}}
\newcommand{\IDOWE}{\text{\sf ID-OWE}}
\newcommand{\INDsIDCCA}{\text{\sf IND-sID-CCA2}}
\newcommand{\INDsIDCPA}{\textsf{IND-sID-CPA}}
\newcommand{\INDHIDCCA}{\text{\sf IND-HID-CCA2}}
\newcommand{\INDHIDCPA}{\text{\sf IND-HID-CPA}}
\newcommand{\indcpad}{\text{\sf IND-CPA$\sf ^D$}\xspace}
\newcommand{\krd}{\text{\sf KR$\sf ^D$}\xspace}
\newcommand{\indcpa}{\text{\sf IND-CPA}\xspace}
\newcommand{\indcca}{\text{\sf IND-CCA}\xspace}
\newcommand{\expr}{\text{\sf Expr}\xspace}
\newcommand{\LWE}{\sf{LWE}}
\newcommand{\RLWE}{\sf{RLWE}}
\newcommand{\musum}{\mu_{\sf{sum}}}
\newcommand{\LSSS}{{\sf LSSS}}
\newcommand{\lsss}{{\sf LSSS}}
\newcommand{\binlsss}{\{0,1\}\mbox{-}{\sf LSSS}}

\newenvironment{gamequote}
               {\list{}{\rightmargin0pt\relax}\item\relax}
               {\endlist}


\newenvironment{myquote}
               {\list{}{\setlength\rightmargin{0pt}}%
                \item[]}
               {\endlist}



\newcommand{\advg}{\mathsf{Adv}_A}

\newcommand{\false}{\mathsf{false}}
\newcommand{\true}{\mathsf{true}}

\newcommand{\remove}[1]{}



\newcommand{\A}{\mathcal{A}}
\newcommand{\B}{\mathcal{B}}

\newcommand{\dash}{\mbox{---}}
\renewcommand{\O}{{\mathcal{O}}}
\newcommand{\Id}{\mathbf{Id}}

\newcommand{\myspan}{\text{span}}
\newcommand{\bool}{\{0,1\}}
\DeclareMathOperator{\rank}{rank}
\DeclareMathOperator{\negl}{negl}
\DeclareMathOperator{\poly}{poly}
\DeclareMathOperator{\Hom}{Hom}
\DeclareMathOperator{\lcm}{lcm}
\DeclareMathOperator{\wt}{wt}


% algorithm names, etc.
\def\id{{\sf id}}
\def\setup{{\sf Setup}}
\def\keygen{{\sf KeyGen}}
%\newcommand{\kgen}{\sf{KeyGen}}
\def\prmsgen{{\sf PrmsGen}}
\def\sig{{\sf Sig}}
\def\sign{{\sf Sign}}
\def\partsign{{\sf PartSign}}
\def\partdecrypt{{\sf PartDec}}
\def\findecrypt{{\sf FinDec}}
\def\combine{{\sf Combine}}
\def\encrypt{{\sf Enc}\xspace}
\def\simenc{{\sf Sim.Enc}}
\def\samenc{{\sf PgmDec}}
\def\decrypt{{\sf Dec}\xspace}
\def\dec{{\sf Dec}\xspace}
\def\enc{{\sf Enc}\xspace}
\def\eval{{\sf Eval}\xspace}
\def\delegate{{\sf Delegate}}
\def\verify{{\sf Verify}}
\def\params{{\sf PP}}
\def\master{{\sf MK}}

\def\sk{{\sf sk}}
\def\evk{{\sf evk}}
\def\pk{{\sf pk}}
\def\rotk{{\sf rotk}}


\def\indpt{{\sf indpt}}
\def\ind{{\sf ind}}
\def\flag{{\sf flag}}
\def\seed{{\sf seed}}
\def\salt{{\sf salt}}
\def\gsalt{{\sf gsalt}}
\def\prcd{{\sf proceed}}
\def\ov{{\sf out}}
\def\trapmode{{\sf Trap\mbox{-}Mode}}
\def\lt{{\sf lessThan}}
\def\mathunderline#1#2{\color{#1}\underline{{\color{black}#2}}\color{black}}
\def\sfek{{\sf ek}}
\def\sfp{{\sf p}}
\def\sfst{{\sf st}}


\newcommand{\hsk}{{\sf{hsk}}}
\newcommand{\sss}{{\mathtt{s}}}

\newcommand{\hpk}{{\sf{hpk}}}
\newcommand{\enckey}{{\sf{enck}}}
\newcommand{\reckey}{{\sf{reck}}}
\newcommand{\relinkey}{{\sf{rlk}}}
\newcommand{\rotkey}{{\sf{rotk}}}
% \newcommand{\key}{{\sf{key}}}

\newcommand{\U}{\ensuremath{\mathcal{U}}}
\newcommand{\Signer}{\ensuremath{\mathcal{S}}}

\newcommand{\Adv}{\mathsf{Adv}}
\newcommand{\iv}{\mathsf{inp}}
\newcommand{\hativ}{\widehat{\iv}}

%\def\PRF{{\sf PRF}}
\def\cPRF{{\sf cPRF}}
\def\RGC{{\sf RGC}}
\def \rgnfa {{\sf RGNFA}}

\def \Hybrid {{\sf {Hybrid}}}

\renewcommand{\setup}{\mathsf{Setup}}
\newcommand{\extract}{\mathsf{KeyGen}}
\renewcommand{\encrypt}{\mathsf{Enc}}
\renewcommand{\decrypt}{\mathsf{Dec}}
\newcommand{\evaluate}{\mathsf{Eval}}
\newcommand{\keydel}{\mathsf{KeyDel}}
\newcommand{\constrain}{\mathsf{Constrain}}

\newcommand{\ct}{{\sf ct}\xspace}
\newcommand{\pt}{\mathsf{pt}}
\newcommand{\fempk}{\MPK}
\newcommand{\femsk}{\MSK}

\newcommand\syme{{\sf SKE}}
\newcommand\symkeygen{{\sf SKE.KeyGen}}
\newcommand\symenc{{\sf SKE.Enc}}
\newcommand\symdec{{\sf SKE.Dec}}
\newcommand\symsk{{\sf K}} %sk_E

\newcommand{\dfampk}{\MPK}
\newcommand{\dfamsk}{\MSK}

\newcommand{\tmmpk}{\MPK}
\newcommand{\tmmsk}{\MSK}
\newcommand{\TMFE}{{\sf TMFE}}
\newcommand{\kTMFE}{{\sf kTMFE}}

\newcommand{\dmfe}{\sf {DFE}} 
\newcommand{\dfe}{\sf{FE}}
\newcommand{\fe}{\sf{1\FE}} %Used in TM-miFE %This is bad notation, \fe should give FE not 1FE. You should try to make notation natural and expressive.
\newcommand{\cfe}{\sf{1\FE}_1} %Used in TMFE
\newcommand{\dcfe}{\sf{1\FE}_2} %Used in TMFE
%================================================
\newcommand{\dfesetup}{\sf {CktFE.Setup}} 
\newcommand{\dfeenc}{\sf {CktFE.Enc}}
\newcommand{\dfekeygen}{\sf {CktFE.KeyGen}}
\newcommand{\dfedec}{\sf {CktFE.Dec}}
\newcommand{\henc}{\mathcal{E}}
\newcommand{\moder}{{\sf mode\mbox{-}real}}
\newcommand{\modet}{{\sf mode\mbox{-}trap}}
\newcommand{\target}{{\sf Target}}

\newcommand{\gbnfa}{\mathsf{RGbNFA}}
\newcommand{\gbnfasetup}{\mathsf{RGbNFA.Setup}}
\newcommand{\gbnfagarble}{\mathsf{RGNfa.Garble}}
\newcommand{\gbnfaenc}{\mathsf{RGNfa.Encode}}
\newcommand{\gbnfaeval}{\mathsf{RGNfa.Eval}}

\newcommand{\gbc}{\mathsf{Gbc}}
\newcommand{\gbcsetup}{\mathsf{RGC.Setup}}
\newcommand{\gbcgarble}{\mathsf{RGC.Garble}}
\newcommand{\gbcenc}{\mathsf{RGC.Encode}}
\newcommand{\gbceval}{\mathsf{RGC.Eval}}

\newcommand{\key}{{\sf key}}
\newcommand{\trap}{{\sf Trap}}
\newcommand{\trapk}{{\sf Trap_K}}
\newcommand{\punc}{\mathsf{Puncture}}

\newcommand{\tmfe}{\sf{TMFE}}
\newcommand{\tmsetup}{\sf {TMFE.Setup}} 
\newcommand{\tmenc}{\sf {TMFE.Enc}}
\newcommand{\tmkeygen}{\sf {TMFE.KeyGen}}
\newcommand{\tmdec}{\sf {TMFE.Dec}}

%\newcommand{\tmmife}{\sf{TM}\mbox{-}\sf{miFE}} %Already defined previously
%\newcommand{\tmmifesetup}{\sf {TM\mbox{-}miFE.Setup}} 
%\newcommand{\tmmifeenc}{\sf {TM\mbox{-}miFE.Enc}}
%\newcommand{\tmmifekeygen}{\sf {TM\mbox{-}miFE.KeyGen}}
%\newcommand{\tmmifedec}{\sf {TM\mbox{-}miFE.Dec}}
\newcommand{\tmmifesetup}{{\sf {kTMFE.Setup}}} 
\newcommand{\tmmifeenc}{{\sf {kTMFE.Enc}}}
\newcommand{\tmmifekeygen}{\sf {kTMFE.KeyGen}}
\newcommand{\tmmifedec}{\sf {kTMFE.Dec}}
%-------------------------------------------

%\newcommand{\dfasetup}{\sf {NfaFE.Setup}} 
%\newcommand{\dfaenc}{\sf {NfaFE.Enc}}
%\newcommand{\dfakeygen}{\sf {NfaFE.KeyGen}}
%\newcommand{\dfadec}{\sf {NfaFE.Dec}}
%\newcommand{\dfasize}{\sf s} 
\newcommand{\dfasize}{\mathsf{s}}
\newcommand{\dfabe}{\mathsf{NfaABE}} 
\newcommand{\dfasetup}{\mathsf{NfaABE.Setup}} 
\newcommand{\dfaenc}{\mathsf{NfaABE.Enc}}
\newcommand{\dfakeygen}{\mathsf{NfaABE.KeyGen}}
\newcommand{\dfadec}{\mathsf{NfaABE.Dec}}

\newcommand{\nfaPe}{\mathsf{NfaPE}^+} 
\newcommand{\nfaPE}{\nfaPe} 
\newcommand{\nfaPEp}{\nfaPe} 

\newcommand{\nfaPEsetup}{\mathsf{NfaPE^+.Setup}} 
\newcommand{\nfaPEenc}{\mathsf{NfaPE^+.Enc}}
\newcommand{\nfaPEEnc}{\mathsf{NfaPE^+.Enc}}

\newcommand{\nfaPEkeygen}{\mathsf{NfaPE^+.KeyGen}}
\newcommand{\nfaPEdec}{\mathsf{NfaPE^+.Dec}}

\newcommand{\nfaFEsize}{\mathsf{s}}
\newcommand{\nfaFEdepth}{\dfadepth}
\newcommand{\dfPEp}{\mathsf{NfaPE}^+} 
\newcommand{\nfaFEsetup}{\mathsf{NfaPE^+.Setup}} 
\newcommand{\nfaFEenc}{\mathsf{NfaPE^+.Enc}}
\newcommand{\nfaFEkeygen}{\mathsf{NfaPE^+.KeyGen}}
\newcommand{\nfaFEdec}{\mathsf{NfaPE^+.Dec}}

\newcommand{\unfabe}{\mathsf{uNfaABE}}
%\newcommand{\ubnfabe}{\mathsf{NfaABE_{(u,b)}}}
\newcommand{\ubnfabe}{\mathsf{NfaABE}}
%\newcommand{\bunfabe}{\mathsf{NfaABE_{(b,u)}}}
\newcommand{\bunfabe}{\mathsf{ABE}}

%\newcommand{\mbuABE}{\mbox{(b,u)-SKABE}} % For introduction
%\newcommand{\mubABE}{\mbox{(u,b)-SKABE}} 
%\newcommand{\muuABE}{\mbox{(u,u)-SKABE}}
%\newcommand{\buABE}{\mathsf{(b,u)\mbox{-}SKABE}}
%\newcommand{\ubABE}{\mathsf{(u,b)\mbox{-}SKABE}}
%\newcommand{\uuABE}{\mathsf{(u,u)\mbox{-}SKABE}} 


\newcommand{\mbuABE}{{\mathsf{ (b,u)\mbox{-}NfaABE}}} % For introduction
\newcommand{\mubABE}{{\mathsf {(u,b)\mbox{-}NfaABE}}} 
\newcommand{\muuABE}{\mathsf{(u,u)\mbox{-}NfaABE}}
\newcommand{\buABE}{\mathsf{(b,u)\mbox{-}NfaABE}}
\newcommand{\ubABE}{\mathsf{(u,b)\mbox{-}NfaABE}}
%\newcommand{\ubABE}{\mathsf{(u,b)\mbox{-}SKABE}}
\newcommand{\uuABE}{\mathsf{(u,u)\mbox{-}NfaABE}} 

\newcommand{\gbdfa}{\mathsf{GbDfa}} 
\newcommand{\gbdfagarble}{\mathsf{GbDfa.Garble}} 
\newcommand{\gbdfaenc}{\mathsf{GbDfa.Enc}}
\newcommand{\gbdfaeval}{\mathsf{GbDfa.Eval}}
\newcommand{\gsk}{{\mathsf{gsk}}}

\newcommand{\uckt}{U_{E}}
\newcommand{\uckti}{U_{\tilde E}}

\newcommand{\afe}{A^{\FE}}


\newcommand{\rlwe}{\mathsf{RLWE}}
%\newcommand{\sivp}{\mathsf{SIVP}}
%\newcommand{\svp}{\mathsf{SVP}}
\newcommand{\rsvp}{\mathsf{RSVP}}
\newcommand{\rsis}{\mathsf{RSIS}}
%\newcommand{\gapsvp}{\mathsf{GapSVP}}

\newcommand{\cktfe}{\sf {CktFE}}
\newcommand{\CktFE}{\cktfe}
\newcommand{\tm}{{\sf TM}}
\newcommand{\dfafe}{\sf {NfaFE}}
\newcommand{\fesim}{\sf {CktFE.Sim}}
\newcommand{\dfasim}{\sf {NfaFE.Sim}}
\newcommand{\tmsim}{\sf {TMFE.Sim}}
\newcommand{\fhe}{\sf {fhe}}

\newcommand{\FE}{\mathsf{FE}}
\newcommand{\LinFE}{\mathcal{FE}_{\mathsf{Lin}}}
\newcommand{\FEapp}{\widetilde{\mathcal{FE}}}%_{\sf{Apx}}}
\newcommand{\FEapprox}{\FEapp}
\newcommand{\FEex}{\mathcal{FE}_{\sf{Exact}}}
\newcommand{\tensor}{\otimes}

\newcommand{\spart}{\mathsf{PART}}
\newcommand{\partsim}{\mathsf{AUG}\mbox{-}\mathsf{SIM}}
\newcommand{\fullsim}{\mathsf{FULL}\mbox{-}\mathsf{SIM}}
\newcommand{\nasim}{\mathsf{NA}\mbox{-}\mathsf{SIM}}
\newcommand{\sasim}{\mathsf{SA}\mbox{-}\mathsf{SIM}}
\newcommand{\selsim}{\mathsf{Sel}\mbox{-}\mathsf{SIM}}
\newcommand{\ssim}{\selsim}


\newcommand{\nausim}{\mathsf{NA}\mbox{-}\mathsf{USIM}}
\newcommand{\naind}{\mathsf{NA}\mbox{-}\mathsf{IND}}
%\newcommand{\ind}{\mathsf{IND}}
\newcommand{\selind}{\mathsf{Sel}\mbox{-}\mathsf{IND}}
\newcommand{\appselind}{\mathsf{Noisy}\mbox{-}\mathsf{Sel}\mbox{-}\mathsf{IND}}
\newcommand{\appind}{\mathsf{Noisy}\mbox{-}\mathsf{IND}}

%\newcommand{\indcpa}{\mathsf{IND}\mbox{-}\mathsf{CPA}}
%\newcommand{\indcca}{\mathsf{IND}\mbox{-}\mathsf{CCA}}
\newcommand{\regpke}{\mathsf{Reg}\mbox{-}\mathsf{PKE}}
\newcommand{\extreg}{\mathsf{Ext}\mbox{-}\mathsf{Reg}}
\newcommand{\pke}{\mathsf{PKE}}
\newcommand{\PKE}{\pke}
\newcommand{\dfadepth}{\mathsf{d}}
\newcommand{\depth}{\mathsf{depth}}
\newcommand{\size}{\mathsf{size}}
\newcommand{\Time}{\mathsf{time}}
\newcommand{\out}{\mathsf{out}}
\newcommand{\PPT}{\mathsf{PPT}}
\newcommand{\NEXT}{\mathsf{Next}}
\newcommand{\AGG}{\mathsf{Agg}}
\newcommand{\SYM}{\mathsf{SYM}}
\newcommand{\SP}{\mathsf{SP}}
\newcommand{\ST}{\mathsf{ST}}
\newcommand{\st}{\mathsf{ST}}
\newcommand{\ACC}{\mathsf{ACC}}
\newcommand{\REJ}{\mathsf{REJ}}
\newcommand{\seedTS}{\mathsf{K^{TS}_\MSK}}
%\newcommand{\seedusr}{\mathsf{K^{User}_\MSK}}
\newcommand{\seedusr}{\mathsf{K^{\AGG}_\MSK}}
%\newcommand{\FTS}{\mathsf{F_{TS}}}
\newcommand{\FTS}{\mathsf{F}} %Defined on 11.9.17
%\newcommand{\cFMSK}{\mathsf{F_{MSK}}}
\newcommand{\cFMSK}{\mathsf{F_{User}}}
\newcommand{\TS}{\mathsf{TS}}
\newcommand{\usr}{\mathsf{User}}
\newcommand{\Eval}{\mathsf{Eval}}
\newcommand{\init}{\mathsf{state\_{init}}}
\newcommand{\ksym}{\mathsf{K_\SYM}}
\newcommand{\kst}{\mathsf{K_\ST}}
\newcommand{\ksp}{\mathsf{K_\SP}}
\newcommand{\kagg}{\mathsf{K_\AGG}}
\newcommand{\knxt}{\mathsf{K_\NEXT}}
\newcommand{\sfH}{\mathsf{H}}
\newcommand{\sfF}{\mathsf{F}}
\newcommand{\type}{\mathsf{type}}
\newcommand{\prfgen}{\mathsf{Gen}}
\newcommand{\fagg}{\mathsf{F}^\AGG}
\newcommand{\fnxt}{\mathsf{F}^\NEXT}
\newcommand{\newlen}{\mathsf{pos}}
\newcommand{\pos}{\mathsf {pos}}
\newcommand{\newpos}{\newlen}
\newcommand{\sfq}{\mathsf{q}}
\newcommand{\stt}{\mathsf{st}}
\newcommand{\sym}{\mathsf{sym}}
\newcommand{\RED}{\mathsf{RED}}
\newcommand{\GREEN}{\mathsf{GREEN}}
\newcommand{\sfi}{\mathsf{i}}
\newcommand{\kstr}{\mathsf{K_{RAND}}}
\newcommand{\evl}{\mathsf{EVAL}}
\newcommand{\sel}{\mathsf{Sel}}
\newcommand{\onect}{\mathsf{OneCT}}
\newcommand{\ad}{\mathsf{Ad}}
\newcommand{\adv}{\mathsf{Adv}}
\newcommand{\YLO}{\mathsf{YELLOW}}
\newcommand{\len}{\mathsf{len}}

\newcommand{\K}{\mathsf{K}}
\renewcommand{\k}{\mathsf{k}}
\newcommand{\hatK}{\widehat{\mathsf{K}}}
\newcommand{\Rand}{\mathsf{R}} % \R was already used.
\newcommand{\hatR}{\widehat{\mathsf{R}}}
%\newcommand{\krnd}{\mathsf{K_{RAND}}}

\newcommand{\rnd}{\mathsf{rand}}
\newcommand{\kfe}{\mathsf{k}\FE} %Used in TM-miFE
\newcommand{\kFE}{\mathsf{k}\FE} %Used in TM-miFE

\newcommand{\rernd}{\mathsf{ReRand}}
\newcommand{\RERND}{\rernd}

\newcommand{\nasimone}{\nasim_{\small one}}
\newcommand{\nasimmany}{\nasim}
\newcommand{\naindone}{\naind_{\small one}}
\newcommand{\naindmany}{\naind}


\newcommand{\adsim}{\mathsf{AD}\mbox{-}\mathsf{SIM}}
\newcommand{\adusim}{\mathsf{AD}\mbox{-}\mathsf{USIM}}
\newcommand{\adind}{\mathsf{AD}\mbox{-}\mathsf{IND}}


\newcommand{\adsimone}{\adsim_{\small one}}
\newcommand{\adsimmany}{\adsim}
\newcommand{\adindone}{\adind_{\small one}}
\newcommand{\adindmany}{\adind}

\newcommand{\tfe}{t\text{-}\mathsf{CktFE}}
\newcommand{\tfesetup}{\tfe\mathsf{.Setup}}
\newcommand{\tfekeygen}{\tfe\mathsf{.Keygen}}
\newcommand{\tfeenc}{\tfe\mathsf{.Enc}}
\newcommand{\tfedec}{\tfe\mathsf{.Dec}}

\newcommand{\fesetup}{\mathsf{CktFE.Setup}}
\newcommand{\fekeygen}{\mathsf{CktFE.Keygen}}
\newcommand{\feenc}{\mathsf{CktFE.Enc}}
\newcommand{\feencoff}{\mathsf{CktFE.EncOff}}
\newcommand{\feencon}{\mathsf{CktFE.EncOn}}
\newcommand{\fedec}{\mathsf{CktFE.Dec}}

\newcommand{\SKFE}{\mathsf{SKFE}}



\newcommand{\mpk}{\mathsf{mpk}}
\newcommand{\msk}{\mathsf{msk}}

\def\PK{\mathsf{pk}}
\def\MPK{\mathsf{MPK}}
\def\MSK{\mathsf{MSK}}
\def\SK{\mathsf{sk}}
\def\felin{\mathsf{FE}_{\mathsf{Lin}}}

%\def\exp{\mathsf{Exp}}
\def\expreal{\exp^{\mathsf{real}}}
\def\expideal{\exp^{\mathsf{ideal}}}
\newcommand{\ps}[2]{\left<#1,#2\right>}


\newcommand{\sis}{{\sf SIS}}
\newcommand{\omisis}{{\sf one}$-${\sf more}$-${\sf ISIS}}
\newcommand{\omlwe}{{\sf one}$-${\sf more}$-${\sf LWE}}
\newcommand{\isis}{{\sf ISIS}}
\newcommand{\lwe}{{\sf LWE}}
\newcommand{\TrapGen}{{\sf TrapGen}}
\newcommand{\ExtBasis}{{\sf ExtBasis}}
\newcommand{\SamplePre}{{\sf SamplePre}}
\newcommand{\SampleGaussian}{{\sf SampleGaussian}}
\newcommand{\lat}{\mathcal{L}}
\newcommand{\lsis}{{\lat}\text{-}{\sf SIS}}
%\newcommand{\encode}[1]{\langle #1 \rangle}
\newcommand{\SampleRight}{{\sf SampleRight}}
% \renewcommand{\Game}{{\sf Game}}
\newcommand{\sivp}{{\sf SIVP}}
\newcommand{\gapsvp}{{\sf GapSVP}}
\newcommand{\hve}{{\sf hve}}
\newcommand{\NIZKPoK}{{\sf NIZKAoK}}
\newcommand{\NIZK}{{\sf NIZK}}
\newcommand{\NIZKAoK}{{\sf NIZKAoK}}
\newcommand{\NIZKAoKDL}{{\sf NIZKAoK_{DL}}}
\newcommand{\NIZKAoKL}{{\sf NIZKAoK_{FHE}}}
\newcommand{\NIZKL}{{\sf NIZK_{FHE}}}
\newcommand{\NIZKAoKSIS}{{\sf NIZKAoK_{SIS}}}
%\newcommand{\NIZKAoK}{{\sf NIZKAoK}}

%\newcommand{\latdistr}[2]{{\mathcal D}_{{#1},{#2}}}



\newcommand{\advant}[3]{\mathrm{{#1}\mbox{-}Adv}{[#2,#3]}}
\newcommand{\rgets}{\ensuremath{\stackrel{\mathrm{R}}{\leftarrow}}}
\newcommand{\abs}[1]{\lvert #1 \rvert}
\newcommand{\round}[1]{\left\lfloor #1 \right\rceil}
\newcommand{\innerprod}[2]{\langle #1,#2 \rangle}
\newcommand{\ceil}[1]{\left\lceil {#1} \right\rceil}
\newcommand{\floor}[1]{\left\lfloor #1 \right\rfloor}
\newcommand{\norm}[1]{\lVert #1 \rVert}
%\newcommand{\GSnorm}[1]{\norm}{{ {#1} }} }
\newcommand{\GSnorm}[1]{\lVert #1 \rVert_{\sf\scriptscriptstyle{GS}}}
\newcommand{\latdistr}[2]{{\mathcal D}_{{#1},{#2}}}
\newcommand{\vconcat}[2]{\frac{#1}{\overline{#2}}}
\newcommand{\Set}[1]{\{ 1,\ldots, #1\} }
\newcommand{\tran}{{\scriptscriptstyle \mathsf{T}}}
\newcommand{\trans}{{\tran}}
\newcommand{\Psibar}{\overline{\Psi}}
\newcommand{\catvect}[2]{\left( \begin{smallmatrix} #1 \\ #2 \end{smallmatrix} \right)}
\newcommand{\simulator}{\mathrm{Sim}}
\newcommand{\Simu}{\mathrm{Sim}}
\newcommand{\simul}{\simulator}
\newcommand{\Simul}{\simul}
\newcommand{\NC}{{\mathsf{NC}}}
%\iflncs
%\else
%\newcommand{\note}[1]{
%\begin{center}
%	\framebox{ \parbox{ 15cm } {\textcolor[rgb]{.8,0,0}{#1}}}
%\end{center}
%}
%\fi
%
%\newcommand{\squish}[1]{\iflncs\vspace{#1} \fi}
%
%\iflncs
%\renewcommand{\paragraph}[1]{\medskip \noindent {\bf #1}}
%%\vspace{-8pt} \subsubsection{#1}}
%\else
%\fi

\renewcommand{\proofname}{{\bf Proof}}

\newenvironment{proof_sketch}{\bigskip
\noindent{\bf Sketch of Proof.} } {\qed\medskip}

\newenvironment{proofof}[1]{\medskip
\noindent{\bf Proof of #1.}}{\qed\medskip}


%\pagestyle{plain}

\parskip=1pt

\newenvironment{MyEnumerate}{
\begin{list}
{\arabic{enumi}.}{
\usecounter{enumi}
% \setlength{\leftmargin}{10pt}%
\setlength{\itemsep}{1pt}
\setlength{\topsep}{3pt} }} {
\end{list}
}

\newenvironment{MyItemize}{
\begin{list}{$\bullet$}{
\usecounter{enumi}
% \setlength{\leftmargin}{12pt}%
\setlength{\topsep}{4pt} \setlength{\itemsep}{4pt} }} {
\end{list}
}

\newcommand{ \publication}[ 1]
{\noindent\vspace*{-1em}\raisebox{19pc}[0pt][0pt]{\hspace*{-0pt}\noindent\parbox[t]{6.5in}{\sl{\begin{center}
#1 \end{center}}}}}



% Assignments
%\def\getsr{\stackrel{\scriptscriptstyle{\$}}{\gets}}
\def\getsr{\gets}
\def\getsd{{:=}}
%\def\bydef{\stackrel{.}{=}}
\def\bydef{\triangleq}
\def\getsf{{\gets}}


% Indistinguishability
\newcommand{\cind}{{\ \stackrel{c}{\approx}\ }}
\newcommand{\compind}{\cind}
\newcommand{\sind}{{\ \stackrel{s}{\approx}\ }}

%%%%%%%%%%%%%%%%


\newcommand{\tfhe}{\mathsf{TFHE}}
\newcommand{\decode}{\mathsf{decode}}

\newcommand{\VK}{{\sf VK}}
\newcommand{\vk}{{\sf vk}}
\newcommand{\dk}{{\sf dk}}
\newcommand{\M}{{\sf M}}
\newcommand{\pp}{{\sf pp}}

\newcommand{\sleft}{\mathsf{\leftarrow}}
\newcommand{\sright}{\mathsf{\rightarrow}}


\newcommand{\ssk}{\mathsf{Sig.sk}}
\newcommand{\svk}{\mathsf{Sig.vk}}
\newcommand{\fhepk}{\mathsf{FHE.PK}}
\newcommand{\fhesk}{{\sf FHE.SK}}
\newcommand{\hepk}{\mathsf{FHE.pk}}
\newcommand{\hesk}{{\sf FHE.sk}}

\newcommand{\sch}{\mathsf{Sch}}
\newcommand{\schsk}{\mathsf{Sch.sk}}
\newcommand{\schvk}{\mathsf{Sch.vk}}
\newcommand{\schkeygen}{\mathsf{Sch.KeyGen}}
\newcommand{\schsign}{\mathsf{Sch.Sign}}
\newcommand{\schverify}{\mathsf{Sch.Verify}}
\newcommand{\RSASign}{\mathsf{RSASign}}



\newcommand{\esk}{\mathsf{PKE.sk}}
\newcommand{\epk}{\mathsf{PKE.pk}}
\newcommand{\tesk}{\mathsf{TEnc.sk}}
\newcommand{\tepk}{\mathsf{TEnc.pk}}
\newcommand{\tepp}{\mathsf{TEnc.pp}}
%newcommand{\ct}{\mathsf{ct}}

\newcommand{\bfclaim}{\textbf{Claim}}
\newcommand{\itproof}{\textit{Proof}}


\newcommand{\bfone}{\mathbf{1}}
\newcommand{\bftwo}{\mathbf{2}}
\newcommand{\bfthree}{\mathbf{3}}
\newcommand{\bffour}{\mathbf{4}}
\newcommand{\bffive}{\mathbf{5}}
\newcommand{\bfsix}{\mathbf{6}}
\newcommand{\bfseven}{\mathbf{7}}
\newcommand{\bfeight}{\mathbf{8}}
\newcommand{\bfnine}{\mathbf{9}}

\newcommand{\ionetoN}{_{i=1}^{N}}
\newcommand{\itwotoN}{_{i=2}^{N}}
\newcommand{\jonetoN}{_{j=1}^{N}}
\newcommand{\jtwotoN}{_{j=2}^{N}}

\newcommand{\expt}{\mathsf{Expt}}
\newcommand{\lrvert}{\lvert\rvert}
\newcommand{\sqbracket}[1]{\left[ #1 \right]}
\newcommand{\rbracket}[1]{\left( #1 \right)}
\newcommand{\Renyi}{R\'enyi }
%\newcommand{\FHE}{\mathsf{FHE}}
\newcommand{\BGGp}{\mathsf{BGG+18}}
\newcommand{\hatsigma}{\hat{\sigma}}
\newcommand{\setj}{\{j\}}
\renewcommand{\ss}{\mathsf{SS}}
\newcommand{\sfspan}{\mathsf{span}}
\renewcommand{\S}{\mathbb{S}}
\newcommand{\sprg}{\mathsf{s_{prg}}}
\newcommand{\kprf}{\mathsf{k_{prf}}}
\newcommand{\sprgi}{\mathsf{s_{i, prg}}}
\newcommand{\sprfi}{\mathsf{sprf}_i}
\newcommand{\sprf}{\mathsf{sprf}}
\newcommand{\PRF}{{\mathsf{PRF}}}
\newcommand{\err}{{\sf err}}
\newcommand{\psig}{{\sf PS}}
\newcommand{\hsprms}{{\sf HS.pp}}
\newcommand{\hspp}{{\sf HS.pp}}
\newcommand{\hssigma}{\pi}
\newcommand{\hspk}{{\sf HS.pk}}
\newcommand{\hssk}{{\sf HS.sk}}
\newcommand{\Hide}{{\sf Hide}}
\newcommand{\Hverify}{{\sf HVerify}}
\newcommand{\Process}{{\sf Process}}
\newcommand{\Sim}{{\sf Sim}}
\newcommand{\s}{{\sf s}}
\newcommand{\sigmam}{{\sigma_m}}
\newcommand{\sigmaM}{{\sigma_M}}
\newcommand{\hkey}{{\sf hkey}}
\newcommand{\sfssk}{\mathsf{ssk}}
\newcommand{\sfsvk}{\mathsf{svk}}
\newcommand{\success}{\mathsf{succ}}
\newcommand{\prms}{\mathsf{prms}}
\newcommand{\accept}{\mathsf{accept}}
\newcommand{\reject}{\mathsf{reject}}
\newcommand{\Query}{{\sf Query}}

\newcommand{\commit}{\mathsf{Commit}}
\newcommand{\com}{\mathsf{com}}
\newcommand{\Genpar}{\mathsf{Gen}} %{\mathsf{Gen_{par}}}
\newcommand{\GenL}{\mathsf{Gen}_\cL}
\newcommand{\ppar}{\mathsf{par}}
\newcommand{\crs}{\mathsf{crs}}
\newcommand{\eps}{\epsilon}

\newcommand{\HVZK}{{\sf HVZK}}
\newcommand{\ufnma}{{\sf UF}\mbox{-}{\sf NMA}}
\newcommand{\ufcma}{{\sf UF}\mbox{-}{\sf CMA}}
\newcommand{\sufcma}{{\sf SUF}\mbox{-}{\sf CMA}}


\newcommand{\hemult}{{\mathsf{FHE.Mult}}}
\newcommand{\hecmult}{{\mathsf{FHE.CMult}}}
\newcommand{\headd}{{\mathsf{FHE.Add}}}
\newcommand{\herot}{{\mathsf{FHE.Rot}}}
\newcommand{\hebts}{{\mathsf{FHE.BTS}}}
\newcommand{\hethinbts}{{\mathsf{FHE.ThinBTS}}}


\newcommand{\bgv}{{\mathsf{BGV}}}
\newcommand{\ctyu}{{\ct_{\vecy_u}}}
\newcommand{\ctmsg}{{\ct_\msg}}
\newcommand{\ctyui}{{\ct_{\vecy_{u, i}}}}
\newcommand{\ctmsgi}{{\ct_{\msg_i}}}
\newcommand{\bin}{\{0,1\}}
\newcommand{\barb}{{\bar b}}
\newcommand{\SIS}{{\sf SIS}}
\newcommand{\zBnd}{\beta}
\newcommand{\ctcorr}{\mathsf{Corr\text{-}ct}}
\newcommand{\notctcorr}{\mathsf{\overline{Corr\text{-}ct}}}
\newcommand{\psk}{\mathsf{psk}}
\newcommand{\ppk}{\mathsf{ppk}}
\newcommand{\ppkset}{\mathcal{PK}}
\newcommand{\Type}{\mathsf{Type}}

\newcommand{\add}{{\sf {Add}}\xspace}
\newcommand{\mult}{{\sf {Mult}}\xspace}
\newcommand{\rot}{{\mathsf{Rot}}}

\newcommand{\tmp}{{\mathsf{tmp}}}
\newcommand{\epoch}{{\mathsf{epoch}}}

\newcommand{\RotSum}{{\mathsf{RotSum}}}
\newcommand{\Replicate}{{\mathsf{Replicate}}}
\newcommand{\InnProd}{{\mathsf{InnProd}}}
\newcommand{\MatVecMult}{{\mathsf{MatVecMult}}}
\newcommand{\VecVecTrsMult}{{\mathsf{VecVecTrsMult}}}
\newcommand{\kPCA}{{\mathsf{kPCA}}}

\newcommand{\ecd}{{\mathsf{Encode}}}
\newcommand{\dcd}{{\mathsf{Decode}}}

\newcommand{\HNF}{{\sf HNF}\xspace}
\newcommand{\svp}{{\sf SVP}\xspace}
\newcommand{\mlwe}{{\sf MLWE}\xspace}
\newcommand{\msis}{{\sf MSIS}\xspace}

\newcommand{\modp}{{\ \mathrm{mod}^+}\ }
\newcommand{\modpm}{\ {\mathrm{mod}^\pm}\ }

\newcommand{\gen}{{\sf gen}}

\newcommand{\Vol}{\mathsf{Vol}}

\def\fmult {\mathrel{\overset{\mbox{\tiny fpt}}{\times}}} %floating point mul


% For IND-games
\makeatletter
\newlength\min@xx
\newcommand*\xxrightarrow[1]{\begingroup
	\settowidth\min@xx{$\m@th\textstyle#1$}
	\@xxrightarrow}
\newcommand*\@xxrightarrow[2][]{
	\sbox8{$\m@th\textstyle#1$}  % subscript
	\ifdim\wd8>\min@xx \min@xx=\wd8 \fi
	\sbox8{$\m@th\textstyle#2$} % superscript
	\ifdim\wd8>\min@xx \min@xx=\wd8 \fi
	\xrightarrow[{\mathmakebox[\min@xx]{\textstyle#1}}]
	{\mathmakebox[\min@xx]{\textstyle#2}}
	\endgroup}
\newcommand*\xxleftarrow[1]{\begingroup
	\settowidth\min@xx{$\m@th\textstyle#1$}
	\@xxleftarrow}
\newcommand*\@xxleftarrow[2][]{
	\sbox8{$\m@th\textstyle#1$}  % subscript
	\ifdim\wd8>\min@xx \min@xx=\wd8 \fi
	\sbox8{$\m@th\textstyle#2$} % superscript
	\ifdim\wd8>\min@xx \min@xx=\wd8 \fi
	\xleftarrow[{\mathmakebox[\min@xx]{\textstyle#1}}]
	{\mathmakebox[\min@xx]{\textstyle#2}}
	\endgroup}
\makeatother % updated
\renewcommand*{\algorithmcfname}{Oracle}

\usetikzlibrary{positioning, arrows.meta}

\title{{\indcpad} and {\krd} security of FHE and application to Threshold-FHE:\\
{\small Attacks Against the \indcpad Security of Exact FHE Schemes.\vspace{-0.6cm}}}

\author{\centering 
Jung Hee Cheon, \underline{Hyeongmin Choe}, Alain Passelègue,\\ Damien Stehlé, Elias Suvanto
}

\date{
\centering\vspace{0.5cm}
Crypto Winter Camp 2024\\
January, 4th, 2024.
}

\begin{document}

%%%%%%%%%%%%%%%%%
% title page
%%%%%%%%%%%%%%%%%
    \begin{frame}[plain]
        \maketitle
    \end{frame}

%%%%%%%%%%%%%%%%%
% Section 1: Definition
%%%%%%%%%%%%%%%%%
\section{\indcpad Security of FHEs:\\ {\normalsize Security Definitions}}

    %%%%%
    % IND-CPA
    %%%%%
    \begin{frame}{IND-CPA security}
    \small
    General security notion for FHE is \textcolor{blue}{\indcpa security}: 
    \vspace{0.5cm}
    \begin{figure}[ht!]
    \centering
    \renewcommand{\arraystretch}{1}
    {\scriptsize
        \begin{tabular}{ccc}
        \underline{\bf \footnotesize $\cA$ (adversary)} & & \underline{\bf \footnotesize $\cC$ (challenger)}\\
        &$\xxleftarrow{1111111111}{\pk}$& $(\sk, \pk) \leftarrow \keygen$, $b \leftarrow \{0,1\}$\\
        \hdashline &&\\
                
        % enc/eval queries
        \enc/\eval with \pk &&\\&&\\
        \hdashline &&\\
        
        Guess bit $b$ & $\xxrightarrow{1111111111}{b}$ & \\
    \end{tabular}}
    \caption{\small IND-CPA security game. \label{fig:indcpa}}
	\end{figure}
    \end{frame}

    %%%%%
    % IND-CPA to IND-CPA^D
    %%%%%
    \begin{frame}{$\sf{IND-CPA^D}$ security?}
    \small
    \indcpad: \indcpa + \textcolor{blue}{\dec} oracle,
    
    \begin{figure}[ht!]
    \centering
    \renewcommand{\arraystretch}{1}
    {\scriptsize
        \begin{tabular}{ccc}
        \underline{\bf \footnotesize $\cA$ (adversary)} & & \underline{\bf \footnotesize $\cC$ (challenger)}\\
        &$\xxleftarrow{1111111111}{\pk}$& $(\sk, \pk) \leftarrow \keygen$, $b \leftarrow \{0,1\}$\\
        \hdashline &&\\
                
        % enc/eval queries
        \enc/\eval with \pk &&\\&&\\
        \hdashline &&\\

        % dec queries
        \textcolor{blue}{\dec query} & \textcolor{blue}{$\xxrightarrow{1111111111}{\ct}$}&\\
        &\textcolor{blue}{$\xxleftarrow{1111111111}{m}$}&\\
        \hdashline &&\\
        
        Guess bit $b$ & $\xxrightarrow{1111111111}{b}$ & \\
    \end{tabular}}
	\end{figure}\pause
    ~~~but only to  \textcolor{blue}{legitimate} {$\ct$}'s. 
    \vspace{0.5cm}\pause
    
    A definitional tool for \textcolor{blue}{legitimate} ciphertext is required:
    \begin{center}
        Shared state: $S \in \left( \mathcal{M} \times \mathcal{M} \times \mathcal{C} \right)^*$
    \end{center}
    \end{frame}

    %%%%%
    % IND-CPA to IND-CPA^D
    %%%%%
    \begin{frame}{$\sf{IND-CPA^D}$ security?}
    \small    
    \begin{center}
        Shared state: $S \in \left( \mathcal{M} \times \mathcal{M} \times \mathcal{C} \right)^*$
    \end{center}
    \begin{figure}[ht!]
    \centering
    \renewcommand{\arraystretch}{1}
    {\scriptsize
        \begin{tabular}{ccc}
        \underline{\bf \footnotesize $\cA$ (adversary)} & & \underline{\bf \footnotesize $\cC$ (challenger)}
        \\
        &$\xxleftarrow{1111111111}{\pk}$ & $(\sk, \pk) \leftarrow \keygen$, $b \leftarrow \{0,1\}$
        \\
        \hdashline &&\\

        % enc queries
        \multirow{1}{*}{$\cO_\enc$} & $\xxrightarrow{1111111111}{\cO_\enc (m_0, m_1)}$ & \multirow{2}{*}{\shortstack{$S.\sf{append}(m_0, m_1, \ct)$\\ where $\ct  = \enc_\pk(m_b)$}}\\
        &$\xxleftarrow{1111111111}{\ct}$ \\
        \hdashline &&\\
        \pause

        % eval queries
        \multirow{2}{*}{$\cO_\eval$} & $\xxrightarrow{1111111111}{\cO_\eval (C, i_1, \cdots, i_k)}$ & \multirow{2}{*}{\shortstack{$S.\sf{append}(m_0^*, m_1^*, \ct^*),$ where\\ 
        $m_i^* = C(S[i_1].m_i, \dots, S[i_k].m_i)$ for $i=0,1$,\\
        $\ct^* \leftarrow \eval_\pk(C, S[i_1].\ct, \cdots, S[i_k].\ct)$}} \\
        &$\xxleftarrow{1111111111}{\ct^*}$ \\
        &&\\
        \hdashline &&\\
                
        Guess bit $b$ & $\xxrightarrow{1111111111}{b}$ & \\
    \end{tabular}}\vspace{-0.2cm}
	\end{figure}
    \vspace{0.5cm}\pause
     ..and \textcolor{blue}{\dec} oracle, which will allow decryption only when \textcolor{blue}{$m_0 = m_1$}.
    \end{frame}
    
    %%%%%
    % IND-CPAD
    %%%%%    
    \begin{frame}{$\sf{IND-CPA^D}$ security}
    \small
    \begin{figure}[ht!]
    \centering
    \renewcommand{\arraystretch}{1}
    {\scriptsize
        \begin{tabular}{ccc}
        \underline{\bf \footnotesize $\cA$ (adversary)} & & \underline{\bf \footnotesize $\cC$ (challenger)}\\
        &$\xxleftarrow{1111111111}{\pk}$& $(\sk, \pk) \leftarrow \keygen$, $b \leftarrow \{0,1\}$\\
        \hdashline &&\\
                
        % enc queries
        \multirow{1}{*}{$\cO_\enc$} & $\xxrightarrow{1111111111}{\cO_\enc (m_0, m_1)}$ & \multirow{2}{*}{\shortstack{$S.\sf{append}(m_0, m_1, \ct)$\\ where $\ct  = \enc_\pk(m_b)$}}\\
        &$\xxleftarrow{1111111111}{\ct}$ & \\
        \hdashline &&\\
        
        % eval queries
        \multirow{2}{*}{$\cO_\eval$} & $\xxrightarrow{1111111111}{\cO_\eval (C, i_1, \cdots, i_k)}$ & \multirow{3}{*}{\shortstack{$S.\sf{append}(m_0^*, m_1^*, \ct^*),$ where\\ 
        $m_i^* = C(S[i_1].m_i, \dots, S[i_k].m_i)$ for $i=0,1$,\\
        $\ct^* \leftarrow \eval_\pk(C, S[i_1].\ct, \cdots, S[i_k].\ct)$}} \\
        &$\xxleftarrow{1111111111}{\ct^*}$ \\
        &&\\
        \hdashline &&\\
        
        % dec queries        
        \multirow{1}{*}{\textcolor{blue}{$\cO_\dec$}} & $\xxrightarrow{1111111111}{\textcolor{blue}{\cO_\dec (j)}}$ & \multirow{2}{*}{\textcolor{blue}{\shortstack{if $S[j].m_0 = S[j].m_1$,\\
        $m \leftarrow \dec_\sk(S[j].\ct)$
        }}}\\
        &\textcolor{blue}{$\xxleftarrow{1111111111}{m}$}\\
        \hdashline &&\\
        
        Guess bit $b$ & $\xxrightarrow{1111111111}{b}$ & \\
    \end{tabular}}
    \caption{\indcpad security game~\cite{EC:LiMic21}. \label{fig:indcpad}}
	\end{figure}
    \end{frame}
    
    %%%%%
    % KRD
    %%%%%    
    \begin{frame}{\krd security}
    \small
    \krd security can be similarly defined, but with ${S} \in (\cM \times \cC)^*$
    \begin{figure}[ht!]
    \centering
    \renewcommand{\arraystretch}{1}
    {\scriptsize
        \begin{tabular}{ccc}
        \underline{\bf \footnotesize $\cA$ (adversary)} & & \underline{\bf \footnotesize $\cC$ (challenger)}\\
        &$\xxleftarrow{1111111111}{\pk}$& $(\sk, \pk) \leftarrow \keygen$\\
        \hdashline &&\\
                
        % enc queries
        \multirow{1}{*}{$\cO_\enc$} & $\xxrightarrow{1111111111}{\cO_\enc (m)}$ & \multirow{2}{*}{\shortstack{$S.\sf{append}(m, \ct)$\\ where $\ct  = \enc_\pk(m)$}}\\
        &$\xxleftarrow{1111111111}{\ct}$ & \\
        \hdashline &&\\
        
        % eval queries
        \multirow{2}{*}{$\cO_\eval$} & $\xxrightarrow{1111111111}{\cO_\eval (C, i_1, \cdots, i_k)}$ & \multirow{3}{*}{\shortstack{$S.\sf{append}(m^*, \ct^*),$ where\\ 
        $m^* = C(S[i_1].m, \dots, S[i_k].m)$,\\
        $\ct^* \leftarrow \eval_\pk(C, S[i_1].\ct, \cdots, S[i_k].\ct)$}} \\
        &$\xxleftarrow{1111111111}{\ct^*}$ \\
        &&\\
        \hdashline &&\\
        
        % dec queries        
        \multirow{1}{*}{\textcolor{blue}{$\cO_\dec$}} & $\xxrightarrow{1111111111}{\textcolor{blue}{\cO_\dec (j)}}$ & \multirow{2}{*}{\textcolor{blue}{
        $m \leftarrow \dec_\sk(S[j].\ct)$}}\\
        &\textcolor{blue}{$\xxleftarrow{1111111111}{m}$} \\
        \hdashline &&\\
        
        Guess bit $b$ & $\xxrightarrow{1111111111}{b}$ & \\
    \end{tabular}}\vspace{-0.2cm}
    \caption{\krd security game~\cite{EC:LiMic21}. \label{fig:indcpad}}
	\end{figure}
    \end{frame}

    %%%%%
    % Li-Micciancio
    %%%%%    
    \begin{frame}{Li-Micciancio~\cite{EC:LiMic21}}
    \small
    Note, for FHEs,
        \begin{itemize}
            \item \indcpad security implies \indcpa security, 
            \item \indcpad security implies \krd security.
        \end{itemize}
    
	Li-Micciancio~\cite{EC:LiMic21}: 
    \begin{itemize}
        \item For exact FHEs, \indcpa security implies \indcpad security.
        \item This is not the case for approximate FHEs: 
        \begin{itemize}
            \item CKKS \krd attack
            \item benefit from additional information (error)
        \end{itemize}
    \end{itemize}

    In the community: 
    \begin{center}
        Exact FHEs, such as BFV/BGV,\\ and DM/CGGI schemes are \indcpad secure!
    \end{center}
    \end{frame}

%%%%%
% Really?
%%%%%

\begin{frame}{}
\begin{center}
    {\Huge \bf Really?}
\end{center} 
\end{frame}

%%%%%%%%%%%%%%%%%
% Section 2: IND-CPA^D Insecurity
%%%%%%%%%%%%%%%%%
\section{\indcpad Security of FHEs:\\ {\normalsize Theoretical Results}}

    %%%%%
    % IND-CPAD
    %%%%%    
    \begin{frame}{What if?}

    
    What if $\dec_\sk(\enc_\pk(m)) \neq m$?

    What if $\dec_\sk(\eval_\pk(C, \ct_1, \cdots, \ct_k))  \neq C(\dec_\sk(\ct_1), \dots, \dec_\sk(\ct_k))$?
    
    \small
    \begin{figure}[ht!]
    \centering
    \renewcommand{\arraystretch}{1}
    {\scriptsize
        \begin{tabular}{ccc}
        \underline{\bf \footnotesize $\cA$ (adversary)} & & \underline{\bf \footnotesize $\cC$ (challenger)}\\
        &$\xxleftarrow{1111111111}{\pk}$& $(\sk, \pk) \leftarrow \keygen$, $b \leftarrow \{0,1\}$\\
        \hdashline &&\\
                
        % enc queries
        \multirow{1}{*}{$\cO_\enc$} & $\xxrightarrow{1111111111}{\cO_\enc (m_0, m_1)}$ & \multirow{2}{*}{\shortstack{$S.\sf{append}(m_0, m_1, \ct)$\\ where $\ct  = \enc_\pk(m_b)$}}\\
        &$\xxleftarrow{1111111111}{\ct}$ & \\
        \hdashline &&\\
    \end{tabular}}
	\end{figure}
    What if $\dec_\sk(\ct) \neq m_b$?
    
    \begin{figure}[ht!]
    \centering
    \renewcommand{\arraystretch}{1}
    {\scriptsize
        \begin{tabular}{ccc}
        
        % eval queries
        \multirow{2}{*}{$\cO_\eval$} & $\xxrightarrow{1111111111}{\cO_\eval (C, i_1, \cdots, i_k)}$ & \multirow{3}{*}{\shortstack{$S.\sf{append}(m_0^*, m_1^*, \ct^*),$ where\\ 
        $m_i^* = C(S[i_1].m_i, \dots, S[i_k].m_i)$ for $i=0,1$,\\
        $\ct^* \leftarrow \eval_\pk(C, S[i_1].\ct, \cdots, S[i_k].\ct)$}} \\
        &$\xxleftarrow{1111111111}{\ct^*}$ \\
        &&\\
        \hdashline &&\\
    \end{tabular}}
    \caption{\indcpad security game~\cite{EC:LiMic21}. \label{fig:indcpad}}
	\end{figure}
    What if $\dec_\sk(\ct^*) \neq C(S[i_1].m_i, \dots, S[i_k].m_i)$?
    
    \end{frame}


	\begin{frame}{Generic \indcpad Attack}
	\begin{figure}[ht]
	\centering
	\renewcommand{\arraystretch}{1}
	{\small
		\begin{tabular}{ccc}
			\underline{\bf Adversary} & & \underline{\bf \small Challenger}\\
			&&\\
			&& $b \leftarrow \{0,1\}$\\
			
			& \hspace{-1.5cm}$\xxrightarrow{11111111111111111}{\cO_\enc (\texttt{T}, \texttt{F})}$ & \hspace*{-.2cm} $S[0] = (\texttt{T}, \texttt{F}, \ct_0)$\\
			
			& \hspace{-1.5cm}$\xxrightarrow{11111111111111111}{\cO_\enc (\texttt{F}, \texttt{F})}$ & \hspace*{-.2cm} $S[1] = (\texttt{F}, \texttt{F}, \ct_1)$\\
			
			& \hspace{-1.5cm}$\xxrightarrow{11111111111111111}{\cO_\eval (\textsf{AND}(\cdot, \cdot), 0, 1)}$ & \hspace*{-.2cm} $S[2] = (\texttt{F}, \texttt{F}, \ct_2)$\\
			
			& \hspace{-1.5cm}$\xxrightarrow{11111111111111111}{\cO_\dec (2)}$ & \hspace*{-.2cm} $m_\textsf{res} \leftarrow \dec_\sk(\ct_2)$\\
			
			& \hspace{-1.5cm}$\xxleftarrow{11111111111111111}{m_\textsf{res} }$ &\\
			
			$b' = 
			\begin{cases}
				0&\text{if } m_\textsf{res} = \texttt{T} \enspace,\\
				1&\text{else}.
			\end{cases}$ &&
	\end{tabular}}
	\caption{Generic and passive \indcpad attack on binary HE. \label{fig:indcpad_bin_1}}
\end{figure}	
	\end{frame}

    %%%%%
    % R/LWE
    %%%%%    
  \hspace*{-0.5cm}
		\begin{tabular}{ccc}
			\underline{\bf Adv} & & \underline{\bf Challenger}\\
			&&\\
			&& ~~~$b \leftarrow \{0,1\}$\\
			& $\xxrightarrow{11111111111111111111}{\enc \text{ query for }m_0=0,\,m_1=1}$ & ~~~$\ct_b \leftarrow \enc(\sk, m_b)$\\
			& $\xxrightarrow{11111111111111111111}{\enc \text{ query for }m=0}$ & ~~~$\ct^* \leftarrow \enc(\sk, m)$\\
			 \hline
			\warning &Possible failure boosting & \warning\\ \hline \\
			& $\xxrightarrow{11111111111111111111}{\Eval \text{ query}~C=AND; \,\ct_b, \,\ct^*}$ & ~~~$\ct_\text{res} \leftarrow \Eval(C; \ct, \ct^*)$\\
			& $\xxrightarrow{11111111111111111111}{\dec \text{ query for }\ct_\text{res}}$ & ~~~$m_\text{res} \leftarrow \dec(\sk, \ct_\text{res})$\\
			& $\xxleftarrow{11111111111111111111}{m_\text{res}}$&\\
			$b' = m_{res}$&&\\
		\end{tabular}
		\begin{block}{Advantage}
			The advantage is the decryption failure probability.
		\end{block}
	\begin{frame}{(R)LWE ciphertext decryption}
	(R)LWE ciphertext structure for $\langle \ct, \sk \rangle = Q \cdot \text{I} + \Delta \cdot \text{m} + \text{e}$:
	\begin{center}
		\begin{tikzpicture}
			% Main rectangle
			\draw (0,0) rectangle (8,1);
			
			% First smaller rectangle with text
			\draw [fill=orange!20, dotted](0.1,0.1) rectangle (0.9,0.9);
			\node at (0.5,0.5) {I};
			
			% First smaller rectangle with text
			\draw [fill=gray!20](1,0) rectangle (3,1);
			\node at (2,0.5) {m};
			
			% Second smaller rectangle with text
			\draw (3,0) rectangle (6,1);
			\node at (3.5,0.5) {};
			
			% Third smaller rectangle with text
			\draw [fill=blue!20](6,0) rectangle (8,1);
			\node at (7,0.5) {e};

            % Q
            \draw [<->] (1, 1.1) -- (8, 1.1);
            \node[above] at (4.5,1.1) {\small $\log_2 Q$};
            
            % Delta
            \draw [<->] (3, -0.1) -- (8, -0.1);
            \node[below] at (5.5,-0.1) {\small $\log_2 \Delta$};
		\end{tikzpicture}
	\end{center}

	Li-Micciancio~\cite{EC:LiMic21}: 
    \begin{itemize}
        \item Any exact homomorphic encryption scheme is \indcpa secure if and only if it is \indcpad secure.
        \item CKKS $KR^D$ security can be explored due to the absence of rounding: if $\Delta \cdot m + e$ is given, recovering $\sk$ is just solving a linear equation $\bmod q$. 
    \end{itemize}
    
    \begin{center}
        \begin{tabular}{|c|c|}
            \hline
            FHE schemse & Rounding after $\langle \ct, \sk \rangle$? \\
            \hline
            BFV/BGV & \checkmark  \\
            \hline
            TFHE/FHEW & \checkmark  \\
            \hline
            CKKS & \xmark \\
            \hline
        \end{tabular}
    \end{center}
	\end{frame}
	
	\begin{frame}{KR$^D$ on BFV}
		Encrypt a fresh ciphertext $\ct_0=$ of zero.
		\begin{center}
			\begin{tikzpicture}
				% Main rectangle
				\draw (0,0) rectangle (10,1);
				
				% First smaller rectangle with text
				\draw [fill=gray!30](1,0) rectangle (3,1);
				\node at (2,0.5) {m=0};
			
				
				% Third smaller rectangle with text
				\draw [fill=blue!20](8,0) rectangle (10,1);
				\node at (9,0.5) {LWE error e};
				\draw[<->] (3,-0.1) -- (10,-0.1);
				\node at (6, -0.5) {$\Delta$};
			\end{tikzpicture}
		\end{center}
	
		Evaluate the ciphertext $\ct_{i+1}=\Eval_{\evk}(Add, i, i)$
		\begin{center}
			\begin{tikzpicture}
				% Main rectangle
				\draw (0,0) rectangle (10,1);
				
				% First smaller rectangle with text
				\draw [fill=gray!30](1,0) rectangle (3,1);
	
				% Third smaller rectangle with text
				\draw [fill=blue!20](4,0) rectangle (6,1);
				\node at (5,0.5) {LWE error e};
				\node[below] at (7,0) {$i$ bits}; % Length label
			\end{tikzpicture}
		\end{center}
	Decrypt $\ct_{\log(\Delta)}$
	\begin{center}
		\begin{tikzpicture}
			% Main rectangle
			\draw (0,0) rectangle (10,1);
			
			% First smaller rectangle with text
			\draw [fill=gray!30](1,0) rectangle (3,1);
		
			% Third smaller rectangle with text
			\draw [fill=blue!20](1.1,0.1) rectangle (2.9,0.9);
			\node at (2,0.5) {LWE error e};
			\node[below] at (7,0) {$\log(\Delta)$ bits}; % Length label
		\end{tikzpicture}
	\end{center}
	\end{frame}

	% \begin{frame}{KR$^D$ on TFHE}
	% 	\begin{figure}
	% 		\centering
	% 		\includegraphics[width=1\linewidth]{tfhebts}
	% 		\caption{TFHE bootstrapping}
	% 		\label{fig:tfhebts}
	% 	\end{figure}
	% \end{frame}
 
	\begin{frame}{erfc}
	Let $e \sim \mathcal{N}(0, \sigma)$ and $t > 0$ a threshold bound, then
		\[Pr[e > t] = \frac{1}{2} \text{erfc} \left(\frac{t}{\sqrt{2} \sigma} \right) \]
	\end{frame}

	\begin{frame}{TFHE Modulus Switch}
		\begin{center}
			\begin{tikzpicture}[node distance=3.5cm,>=latex']
				
				\tikzstyle{block} = [rectangle, draw, text centered, text width=3cm, minimum height=1.5cm, node distance=4cm, rounded corners]
				\tikzstyle{arrow} = [->,>=latex']
				
				\node [block, fill=blue!30] (lwe) {LWE ($q=2^{64}$)\\ Var: $\sigma^2$};
				\node [block, right of=lwe, fill=orange!30, rounded corners] (bootstrap) {Modulus\\ Switch};
				\node [block, right of=bootstrap, fill=blue!30] (lwebr) {LWE ($q=2\cdot N$)\\Var: $\sigma^2 + \sigma_{MS}^2$};
				
				\draw [arrow] (lwe) -- (bootstrap);
				\draw [arrow] (bootstrap) -- (lwebr);
				\draw [<-,>=latex'] (bootstrap.north) -- ++(0,0.3cm) node[above, text width=3cm, align=center] {$\vec{e}_{MS} ~\in~[-2^{-2N-1}, 2^{-2N-1}]^{n+1}$};
				
			\end{tikzpicture}
		\end{center}
		The vector $\vec{e}_{MS}$ is $\textbf{public data}$ (computable from ciphertext).
		\[e_{MS} := \langle \vec{e}_{MS}, \sk \rangle\]
		The ciphertext is still correct with probability $pr_{MS}$ if we have
		\[c_{MS} \cdot \sqrt{\sigma^2 + \sigma_{MS}^2} < \frac{\Delta}{2}\]
		TFHE-rs uses $c_{MS} = 7.2$ by default.
		\[erfc(\frac{7.2}{\sqrt{2}}) = 2^{-40}\]
	\end{frame}

	
	\begin{frame}{$KR^D$ attack on TFHE}
		\begin{block}{Decryption failure}
			The Modulus Switch corrupts the message of the input LWE ciphertext with error $e$ if
			\[e+e_{MS} \geq \frac{\Delta}{2}\]
		\end{block}
		\begin{block}{Primary source of decryption failure}
			Other parts of PBS don't corrupt the underlying message of ciphertext with such a high probability.
		\end{block}
		\begin{block}{Estimating the security loss with such hints}
			We fully recover the secret key of TFHE-rs (default parameters) with less than 256 decryption failures. How?
		\end{block}
	\end{frame}

	\begin{frame}{Distributions of rounding error depending on corresponding secret key}
		\begin{block}{Lemma}
			Let~$\sigma,t>0$ and a secret key $s \in \{0,1\}^n$. Let $e_{MS} \sim \mathcal{U}([-1/2, 1/2]^n)$ 
			and~$e_{preMS} \sim \mathcal{N}(0, \sigma^2)$. Let~$i \leq n$ and~$Y_i = \langle e_{MS}, s \rangle - e_{MS, i} s_i$.
			The probability density function~$f$ of~$e_i$ conditioned on the event $\langle e_{MS}, s \rangle + e_{preMS} > t$ (i.e. failure) satisfies, for all~$x_i \in [-1/2,1/2]$: 
			\[
			f(x_i) = \left\{
			\begin{array}{ll}
				\frac{\Pr[x_i + Y_i + e_{preMS} > t]}{\Pr[  e_i + Y_i + e_{preMS} > t]}& \text{if } s_i=1, \\
				1& \text{if } s_i = 0.
			\end{array}\right.\]
		\end{block}
	\end{frame}

	% \begin{frame}{Distribution of public $e_{MS, i}$ conditioned on $s_i$}
	% 	\begin{figure}[h]
	% 		\centering

	% 		\caption{Distributions of the coefficients of $e$ conditioned on decryption failures, for the TFHE-rs library. The blue and orange curves respectively  depict the pdfs of~$e_{MS, i}$ when~$s_i=1$ and~$s_i=0$.
	% 		}
	% 		\includegraphics[width=0.7\linewidth]{krd_core.png}
	% 	\end{figure}
	% \end{frame}

	% \begin{frame}{KRD pseudocode}
	% 	\begin{figure}
	% 		\centering
	% 		\includegraphics[width=0.8\linewidth]{tkrd}
	% 		\caption{The output for each $i \leq n$ is incorrect with probability $O(\text{erfc}(\sqrt{f} \alpha))$}
	% 		\label{fig:tkrd}
	% 	\end{figure}
	% \end{frame}

	% \begin{frame}{Experimental results}
	% 	\begin{figure}
	% 		\centering
	% 		\includegraphics[width=0.7\linewidth]{exp}
	% 		\caption{Performance of the attack as a function of the number of decryption failures. The cyan and magenta curves respectively correspond to the custom parameter set and the TFHE-rs default parameters.}
	% 		\label{fig:tkrd}
	% 	\end{figure}
	% \end{frame}

	\section{\indcpad security}
	
		\begin{frame}{$\text{IND-CPA}^D$ security definition}
		\vspace*{-0.3cm}
		Shared state \textbf{S} consisting of tuples in $\mathcal{M} \times \mathcal{M} \times \mathcal{C}$:
		\begin{algorithm}[H]
			\scriptsize
			\caption{Encryption oracle $\cO_\enc^\indcpad(\pk, m_0, m_1)$}
			$\ct \gets \enc_{\pk}(m_\mathbf{b})$\\
			$S.append(m_0, m_1, \ct)$\\
			\Return{$\ct$}
		\end{algorithm}
		\begin{algorithm}[H]
			\scriptsize
			\caption{Evaluation oracle $\cO_\Eval^\indcpad(g, i_1, \dots, i_k)$}
			$\ct \gets \Eval_{\evk}(g, S[i_1].\ct, \dots, S[i_k].\ct)$\\
			$r_0 := g(S[i_1].m_0, \dots, S[i_k].m_0)$\\
			$r_1 := g(S[i_1].m_1, \dots, S[i_k].m_1)$\\
			$S.append(r_0, r_1, \ct)$\\
			\Return{$\ct$}
		\end{algorithm}
		
		\begin{algorithm}[H]
			\scriptsize
			\caption{Decryption oracle $\cO_\dec^\indcpad(\sk, j)$}
			\If{$S[j].m_0 = S[j].m_1$}{
				$m \gets \dec_\sk(S[j].\ct)$\\
				\Return{$m$}
			}
		\end{algorithm}
		Task: Recover $\mathbf{b}$
	\end{frame}

	\begin{frame}{Trivial attack}
		\hspace*{-0.5cm}
		\begin{tabular}{ccc}
			\underline{\bf Adv} & & \underline{\bf Challenger}\\
			&&\\
			&& ~~~$b \leftarrow \{0,1\}$\\
			& $\xxrightarrow{11111111111111111111}{\enc \text{ query for }m_0=0,\,m_1=1}$ & ~~~$\ct_b \leftarrow \enc(\sk, m_b)$\\
			& $\xxrightarrow{11111111111111111111}{\enc \text{ query for }m=0}$ & ~~~$\ct^* \leftarrow \enc(\sk, m)$\\
			 \hline
			\warning &Possible failure boosting & \warning\\ \hline \\
			& $\xxrightarrow{11111111111111111111}{\Eval \text{ query}~C=AND; \,\ct_b, \,\ct^*}$ & ~~~$\ct_\text{res} \leftarrow \Eval(C; \ct, \ct^*)$\\
			& $\xxrightarrow{11111111111111111111}{\dec \text{ query for }\ct_\text{res}}$ & ~~~$m_\text{res} \leftarrow \dec(\sk, \ct_\text{res})$\\
			& $\xxleftarrow{11111111111111111111}{m_\text{res}}$&\\
			$b' = m_{res}$&&\\
		\end{tabular}
		\begin{block}{Advantage}
			The advantage is the decryption failure probability.
		\end{block}
	\end{frame}
	
	\begin{frame}{(In)security of Threshold FHE (foretaste)}
		\begin{itemize}
			\item $\Pi$ Treshold FHE: (Setup, Enc, Eval, \textbf{PDec}, \textbf{FinDec})
			\item $\Pi^*$ FHE scheme: (Setup, Enc, Eval, \textbf{Dec}) where $\textbf{Dec}= \Pi.\text{FinDec}_{pk} (\{\Pi.\text{PDec}_{sk_i}(\cdot)\})$
		\end{itemize}
	
		\[\Pi \text{ Threshold-Secure} \Rightarrow \Pi^* \text{ indcpa}^D \text{ secure}\]
		N.B. Noah's Ark partial decryption can fail even if homomorphic evaluation is perfectly correct.
	\end{frame}

 \section{Summary}
    %%%%%
    % Our Result
    %%%%%    
    \begin{frame}{Our Result}
    \small
    Theoretical results: 
    \begin{itemize}
        \item Exact FHE schemes are also \indcpad-insecure \textcolor{blue}{unless} the correctness is $1-\operatorname{negl}(\lambda)$\footnote{During the whole homomorphic operations}.
        \item Correctness of any Threshold FHE should be $1-\operatorname{negl}(\lambda)$ since, 
        \begin{center}
            {\sf Thresh-}\indcpa security of $\Pi$ $\Rightarrow$ \indcpad security of $\Pi^*$,
        \end{center}
        where $\Pi^*$ is the underlying FHE scheme of $\Pi$ (Threshold FHE scheme).
    \end{itemize}\vspace{0.3cm}
    
    Practical results: \krd attacks on
        \begin{itemize}
            \item BFV/BGV
            \item CGGI
            \item CGGI-based Threshold FHE\footnote{Noah's Ark}
        \end{itemize}\vspace{0.3cm}
    \end{frame}
    
    %%%%%
    % Thank You
    %%%%%    
    \begin{frame}{}
    \begin{center}
        {\Huge\bf Thank You!}
    \end{center}
    \end{frame}
\end{document}
